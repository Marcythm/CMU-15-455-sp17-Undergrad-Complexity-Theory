\special{dvipdfmx:config z 0}
\documentclass{article}
\usepackage{marcythm}

\title{Undergraduate Complexity Theory \\ Lecture 17: Savitch's Theorem \& NL}
\author{Marcythm}
% \date{\today}
\date{July 20, 2022}

\begin{document}
\maketitle{}

\section{Lecture Notes}

\begin{theorem}[Savitch's Theorem]
  For any function \(f: \N \to \R^+\), where \(f(n) \geq n\),
  \[ \NSPACE(f(n)) \subseteq \SPACE(f^2(n)) \]
\end{theorem}

idea of Savitch's Theorem: ``Middle-first search''.

\begin{algorithm}
  \begin{algorithmic}
    \Procedure{Path?}{$x, y, k$} \Comment{whether there is a path from $x$ to $y$ within $2^k$ steps}
      \If{$k = 0$}
        \State \Return truth value of $x = y$
      \Else
        \For{$w \in V$}
          \If{\Call{Path?}{$x, w, k - 1$} \&\& \Call{Path?}{$w, y, k - 1$}}
            \State \Return true
          \EndIf
        \EndFor
      \EndIf
      \State \Return false
    \EndProcedure
  \end{algorithmic}
\end{algorithm}

need to store \(\log n\) stack variables (the depth of recursion), each with \(O(\log n)\) space, so \(O(\log^2 n)\) space in total. time complexity: \(O(n^k)\).

\begin{definition}[Nondeterminism-based definition of \NL]
  \( \NL = \NSPACE(\log n)\).
\end{definition}

\begin{proposition}
  \( \prob{ST-PATH} \in \NL \).
\end{proposition}

\begin{proof}
  Nondeterministically choose each step, and maintain a counter for length.
\end{proof}

\begin{theorem}
  \( \NL \subseteq \P, \NL \in \SPACE(\log^2 n)\).
\end{theorem}

\begin{proof}
  see reading section.
\end{proof}

\section{Reading}

\subsection{sipser 8.4 (The Classes \L and \NL)}

\begin{quote}
  If \(M\) is a TM that has a separate read-only input tape and \(w\) is an input, a configuration of \(M\) on \(w\) is a setting of the state, the work tape, and the positions of the two tape heads. The input \(w\) is not a part of the configuration of \(M\) on \(w\).
\end{quote}

Thus total number of configurations of \(M\) on \(w\) is \(|Q| n f(n) |\Gamma|^{f(n)}\), i.e. \(n2^{O(f(n))}\), can extend Savitch's Theorem to \(f(n) \geq \log n\).

\end{document}
