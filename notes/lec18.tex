\special{dvipdfmx:config z 0}
\documentclass{article}
\usepackage{marcythm}

\title{Undergraduate Complexity Theory \\ Lecture 18: \NL-Completeness and Logspace Reductions}
\author{Marcythm}
% \date{\today}
\date{July 20, 2022}

\begin{document}
\maketitle{}

\section{Lecture Notes}

\begin{theorem}
  For \(f(n) \geq \log n\), \( \NSPACE(f(n)) \subseteq \TIME(2^{O(f(n))}) \).
\end{theorem}

\begin{theorem}
  For \(f(n) \geq \log n\), \( \NSPACE(f(n)) \subseteq \SPACE(f(n)^2) \).
\end{theorem}

\begin{corollary}
  \( \NPSPACE \subseteq \PSPACE \).
\end{corollary}

\begin{corollary}
  \( \NPSPACE = \PSPACE \).
\end{corollary}

\begin{claim}
  \prob{ST-PATH} is ``\NL-complete''.
\end{claim}

How to define sensible reduction for logspace? \(\leq_m^P\) is a bad choice, since \L is not closed under it.

Ideal: \(A \in \NL, A \leq B, B \in \L\), then \(A \in \L\).

idea: When using reductions, must be as weak as the weakest class you care about.

Need to define \(leq_m^L\), ``log-space reduction'', want:

\begin{enumerate}
  \item closure: \(A \leq_m^L B, B \in \L \implies A \in \L\), and \(B \in \NL \implies A \in \NL\).
  \item transitivity: \(A \leq_m^L B, B \leq_m^L C \implies A \leq_m^L C\).
\end{enumerate}

\begin{definition}
  \(A \leq_m^L B\) if \(\exists R: \setof{0, 1}^* \to \setof{0, 1}^*\) computable (write-once output tape) in \(O(\log n)\) space s.t. \(\forall x: x \in A \iff R(x) \in B\).
\end{definition}

model for space-bounded computation w/ output: read-only input tape, space-bounded (\(O(\log n)\)) r/w work tapes, write-once output tape.

\begin{theorem}
  \prob{ST-PATH} is ``\NL-complete''.
\end{theorem}

\begin{proof}
  \( \prob{ST-PATH} \in \NL \).

  Let \(A \in \NL\), we need to show \(A \leq_m^L \prob{ST-PATH}\).
  Say \(N\) is an \(O(\log n)\)-space nondeterministic TM deciding \(A\).
  Claim: exists a deterministic \(O(\log n)\)-space-computable \(R: \setof{0, 1}^* \to \setof{0, 1}^*\) that, given \(x \in \Sigma^n\), outputs config graph \(G_{N, x}\) and \(C_{\it start}, C_{\it acc}\).

  Thus, \(x \in A \iff N(x) \text{ acc} \iff \exists\) path \(C_{\it start} \to C_{\it acc} \) in \(G_{N, x} \iff R(x) \in \prob{ST-PATH}\).
\end{proof}

\begin{theorem}
  If \(P, Q: \setof{0, 1}^* \to \setof{0, 1}^*\) are computable in \(O(\log n)\) space, then so is \(R(x) = Q(P(x))\).
\end{theorem}

core technique: recalculate \(P(x)\) every time it is acquired by \(Q\) to save space.

\begin{corollary}[closure]
  \(A \leq_m^L B, B \in \L \implies A \in \L\).
\end{corollary}

\begin{corollary}[transitivity]
  \(A \leq_m^L B, B \leq_m^L C \implies A \leq_m^L C\).
\end{corollary}

\begin{exercise}
  \(A \leq_m^L, B \in \NL \implies A \in \NL\).
\end{exercise}

\section{Reading}

\subsection{sipser 8.5 (\NL completeness)}

definition of log space reducibility

log space reducibility implies poly-time reducibility

\begin{theorem}
  \(A \leq_m^L B, B \in \L \implies A \in \L\).
\end{theorem}

\begin{theorem}
  \prob{ST-PATH} is \NL-complete.
\end{theorem}

\begin{corollary}
  \( \NL \subseteq \P \).
\end{corollary}

\end{document}
