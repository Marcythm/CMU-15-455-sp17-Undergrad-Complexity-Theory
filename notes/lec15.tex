\special{dvipdfmx:config z 0}
\documentclass{article}
\usepackage{marcythm}

\title{Undergraduate Complexity Theory \\ Lecture 15: \coNP}
\author{Marcythm}
% \date{\today}
\date{July 18, 2022}

\begin{document}
\maketitle{}

\section{Lecture Notes}

idea: \NP: efficiently certifying \(x \in L\), \coNP: efficiently certifying \(x \notin L\). Recall \prob{UNSAT} in hw5.

\begin{definition}
  \( \coNP = \setof{L : \ol{L} \in \NP} \).
\end{definition}

\begin{remark}
  \( \coNP \neq \ol{NP} \).
\end{remark}

\begin{theorem}
  \( \prob{SAT} \in \P \implies \prob{UNSAT} \in \P \).
\end{theorem}

\begin{theorem}
  \( A \leq_m^\P B \iff \ol{A} \leq_m^\P \ol{B} \).
\end{theorem}

\begin{theorem}
  \P is closed under complement.
\end{theorem}

\begin{theorem}
  \( \P \subseteq \coNP \).
\end{theorem}

\begin{theorem}
  \( \P = \NP \implies \P = \coNP \).
\end{theorem}

\begin{corollary}
  \( \P = \NP \implies \coNP = \NP \).
\end{corollary}

\begin{corollary}
  \( \coNP \neq \NP \implies \P \neq \NP \).
\end{corollary}

\begin{theorem}
  \prob{UNSAT} is \coNP-complete.
\end{theorem}

\begin{proof}
  \( \forall A \in \coNP: \ol{A} \in \NP \implies A \leq_m^\P \ol{A} \leq_m^\P \prob{SAT} \leq_m^\P \prob{UNSAT} \).
\end{proof}

\begin{definition}
  \( \prob{TAUTOLOGY} = \setof{ \encoding{\phi} : \text{every truth assignment makes \(\phi\) true}} \).
\end{definition}

\( \prob{TAUTOLOGY} \in \NP \)? \( \prob{TAUTOLOGY} \in \coNP \)? \( \ol{\prob{TAUTOLOGY}} \in \NP \implies \prob{TAUTOLOGY} \in \coNP \).

\( \prob{PRIME} \in \coNP \).

review:
\begin{enumerate}
  \item \( L \in \NP \): \(\forall x \in L\), \(\exists\) succinct efficiently checkable proof of \(x \in L\).
  \item \( L \in \coNP \): \(\forall x \notin L\), \(\exists\) succinct efficiently checkable proof of \(x \notin L\).
  \item \( L \in \NP \cap \coNP \): \ldots, has ``good characterization''.\ e.g.

  \begin{enumerate}
    \item \prob{PERFECT-MATCHING}, obviously in \NP.
    Suppose the graph \(G = (L, R, E)\), the Hall's Theorem: \(\forall S \subseteq L: |N(S)| \geq |S|\) implies \(G\) has PM, which is the converse of the intuition: \(\exists S \subseteq L: |N(S)| < |S|\) implies \(G\) has no PM. Then also \(\prob{PERFECT-MATCHING} \in \coNP\). Actually, \( \prob{PERFECT-MATCHING} \in \P \). \\
    \item A similar question: \(\prob{LinearProgramming} \in \NP \cap \coNP\), whether it's in \P? unknown til now. \\
    \item \( \prob{PRIMES} \in \NP \) is shown in 1975 by Pratt, thus it's also in \( \NP \cap \coNP \). It's proven in \P. \\
    \item \( \prob{FACTOR} \in \NP \cap \coNP\), here \( \prob{FACTOR} = \setof{\encoding{X, A, B}: \text{\(X\) has a prime factor between \(A\) and \(B\)}} \).
  \end{enumerate}
\end{enumerate}

\begin{theorem}
  \(B\) is prime iff \(\exists A \in [1, B)\) s.t. \(A, A^2, A^3, \ldots, A^{B-2} \neq 1 \pmod{B}\).
\end{theorem}

\end{document}
