\special{dvipdfmx:config z 0}
\documentclass{article}
\usepackage{marcythm}

\title{Undergraduate Complexity Theory \\ Lecture 23: The Polynomial Hierarchy}
\author{Marcythm}
% \date{\today}
\date{July 25, 2022}

\begin{document}
\maketitle{}

\section{Lecture Notes}

motivation:

recall: \NP and \coNP

\begin{observation}
  \(\P = \NP \implies \NP = \coNP\) (contra: \(\NP \neq \coNP \implies \P \neq \NP\))
\end{observation}

\(\prob{ECLIQUE} = \setof{\encoding{G, k}: \text{\(G\) is a graph with a largest clique of exactly size \(k\)}}\)

\(\prob{SMALLEST-CIRCUIT} = \setof{\encoding{C}: \text{\(C\) is the smallest circuit computing the function \(C_f\) that \(C\) computes}}\)

\begin{notation}
  \(Q^{n} u\) means \(Q u, |u| \leq m\) where \(Q \in \setof{\exists, \forall}\).
\end{notation}

\begin{definition}
  \(L \in \Sigma_2\) if \(\exists\) poly-time TM \(V\), \(\exists\) polynomial \(p\) s.t.
  \[ x \in L \iff \exists^{p(x)} u_1 \forall^{p(x)} u_2 : V(x, u_1, u_2) = 1 \]
\end{definition}

\begin{definition}
  \(L \in \Pi_2\) if \(\exists\) poly-time TM \(V\), \(\exists\) polynomial \(p\) s.t.
  \[ x \in L \iff \forall^{p(x)} u_1 \exists^{p(x)} u_2 : V(x, u_1, u_2) = 1 \]
\end{definition}

\begin{observation}
  \(\Pi_2 = \co\Sigma_2\).
\end{observation}

\begin{claim}
  \(\prob{ECLIQUE} \in \Sigma_2\).
\end{claim}

\begin{exercise}
  \(\prob{SMALLEST-CIRCUIT} \in \Pi_2\).
\end{exercise}

\begin{notation}
  \(\Sigma_0 = \Pi_0 = \P, \Sigma_1 = \NP = \exists \P, \Pi_1 = \coNP = \forall \P, \ldots, \Sigma_i = \exists \Pi_{i-1}, \Pi_i = \forall \Sigma_{i-1}\).
\end{notation}

\begin{observation}
  \(\Sigma_i \subseteq \Sigma_{i+1} \cap \Pi_{i+1}\), \(\Pi_i \subseteq \Pi_{i+1} \cap \Sigma_{i+1}\).
\end{observation}

\begin{definition}[Polynomial Hierarchy]
  \(\PH = \bigcup_{k \in \N} \Sigma_k = \bigcup_{k \in \N} \Pi_k\) contains those languages can be described by constant number of quantifiers.
\end{definition}

\begin{claim}
  \(\PH \subseteq \PSPACE\).
\end{claim}

\begin{theorem}
  \(\P = \NP \implies \P = \PH\).
\end{theorem}

\begin{theorem}
  \(\Sigma_i = \Pi_i \implies \Sigma_i = \Pi_i = \PH\). (hierarchy ``collapses'' to the \(i\)th level)
\end{theorem}

Complete problems for \(\Sigma_i, \Pi_i, \PH\):
Let \(\varphi(y_1, y_2, \ldots, y_i)\) be a boolean formula where each \(y_j\) is a vector (or sequence) of boolean variables.

\begin{definition}
  \(\Sigma_i\prob{-SAT} = \setof{\encoding{\varphi(y_1, y_2, \ldots, y_i)} : \exists z_1 \forall z_2 \cdots \varphi(z_1, z_2, \ldots, z_i) = 1}\)
\end{definition}

\begin{exercise}
  \(\Sigma_i\prob{-SAT}\) is \(\Sigma_i\)-complete.
\end{exercise}

the same for \(\Pi_i\).

\begin{claim}
  If \(\exists L\) s.t. \(L\) is \PH-complete, then \(\exists i\) s.t. \(\PH = \Sigma_i\).
\end{claim}

\section{Reading}

\subsection{sipser 10.3 (Alternation)}

definition of Alternating Turing Machine, \ATIME, \ASPACE, \AP, \APSPACE, \AL

\begin{theorem}
  For \(f(n) \geq n\), we have \(\ATIME(f(n)) \subseteq \SPACE(f(n)) \subseteq \ATIME(f^2(n))\). \\
  For \(f(n) \geq \log n\), we have \(\ASPACE(f(n)) = \TIME(2^{O(f(n))})\).
\end{theorem}

\begin{corollary}
  \(\AL = \P, \AP = \PSPACE, \APSPACE = \EXP\).
\end{corollary}

definition of \PH.

\end{document}
