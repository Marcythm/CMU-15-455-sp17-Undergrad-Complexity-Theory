\special{dvipdfmx:config z 0}
\documentclass{article}
\usepackage{marcythm}

\title{Undergraduate Complexity Theory \\ Lecture 8: NP}
\author{Marcythm}
% \date{\today}
\date{July 12, 2022}

\begin{document}
\maketitle{}

\section{Lecture Notes}

\begin{conjecture}
  \( \P \neq \NP ( \iff \prob{3-SAT} \notin \P ) \)
\end{conjecture}

\begin{conjecture}[Exponential Time Hypothesis]
  \( \exists \delta>0: \prob{3-SAT} \notin \TIME((1+\delta)^n)\)
\end{conjecture}

\begin{conjecture}[Strong ETH]
  \( \forall \delta>0: \exists k: \prob{k-SAT} \notin \TIME((2-\delta)^n) \)
\end{conjecture}

\( \mathrm{SETH} \implies \mathrm{ETH} \implies \P \neq \NP \).

\begin{theorem}[ABV '15]
  SETH implies \( \forall \eps>0 \), cannot solve \prob{LCS} in time \(O(n^{2-\eps})\).
\end{theorem}

idea of \NP: for many problems \(\exists\) brute force algo, enumerate (exp many) ``candidate sol''s and check in poly-time if each is a genuine sol.\ e.g. \prob{ST-PATH}, \prob{HAMILTONIAN-PATH}, \prob{3-COL}, \prob{CIRCUIT-SAT}, \prob{COMPOSITE}. \\

two features:
\begin{enumerate}
  \item[0.] ``candidate sol''s  are encodable by strings with polynomial length.
  \item[1.] \(\exists\) poly-time algo to \ul{check} if a candidate sol is a genuine sol.
\end{enumerate}

Informally, a problem is in \NP if a checking algo exists. ``candidate sol'' are also called ``potential sol'', ``witness'', ``certificate'', etc. A problem highly believed not in \NP: \prob{UN-3COL}.

\begin{definition}
  An algorithm(TM) \(V\) is a {\it verifier for language \(L\)} if
  \begin{enumerate}
    \item \(V\) takes as input a pair \(\encoding{x, y}\)
    \item \(\forall x: x \in L \iff \exists y: V(\encoding{x, y})\) accepts.
  \end{enumerate}
\end{definition}

\begin{definition}
  Verifier \(V\) is said to be {\it ``polynomial time''} if \(V(\encoding{x, y})\) runs in time \(O(|x|^k)\) for some \(k \in \N\).
\end{definition}

\begin{remark}
  Subtlety: \(V\)'s runtime is measured in terms of \(|x|\).
\end{remark}

\begin{definition}
  \( \NP = \setof{L : \text{\(L\) has a poly-time verifier}} \)
\end{definition}

e.g. \( \prob{SQUARES} = \setof{\encoding{B} : B \in \N, \exists x \in \N: x^2 = B } \in \NP \). \( \prob{3COL} \in \NP \).

Subtlety in verifier of \prob{SQUARE}: mark the input as \(\encoding{x, y}\), after interpreting \(x = \encoding{B}\), only read first \(|x|\) symbols of \(y\). If \(|y| > |x|\), then reject.

\begin{theorem}
  \( \P \subseteq \NP \).
\end{theorem}

\begin{proof}
  Let \(L \in \P\), thus \(\exists\) a poly-time TM \(M\) s.t. \(x \in L \iff M(x) \) accepts.
  Consider the TM \(V\) that runs on input \(\encoding{x, y}\): Do \(M(x)\), then \ldots
  \begin{enumerate}
    \item Claim \(V\) is poly-time verifier. \ldots
    \item Claim \(V\) verifies \(L\). \ldots
  \end{enumerate}
\end{proof}

About \P = \NP, upcoming: Cook-Levin Theorem: \( \P = \NP \iff \prob{3SAT} \in \P \).

\section{Reading}

\subsection{Sipser 7.3 (The Class \NP)}

definition of verifier, \NP (by verifiers), and \NTIME.

\end{document}
