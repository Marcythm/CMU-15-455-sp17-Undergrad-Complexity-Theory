\special{dvipdfmx:config z 0}
\documentclass{article}
\usepackage{marcythm}

\title{Undergraduate Complexity Theory \\ Lecture 10: Reductions}
\author{Marcythm}
% \date{\today}
\date{July 13, 2022}

\begin{document}
\maketitle{}

\section{Lecture Notes}

\begin{theorem}
  If \(\prob{SAT} \in \P\), then \(\prob{4COL} \in \P\).
\end{theorem}

idea: given \(M_{\prob{SAT}}\) solving \prob{SAT} in poly-time, can build algo \(M_{\prob{4COL}}\) solving \prob{4COL} in poly-time, using \(M_{\prob{SAT}}\) as a subroutine. The whole idea is denoted as \( \prob{4COL} \leq^P \prob{SAT} \), which means \prob{4COL} is at most polynomial slower than \prob{SAT}.

\begin{definition}
  Let \(A, B\) be languages, \(A\) has a poly-time {\bf mapping reduction} to \(B\), written \(A \leq_m^P B\), if exists poly-time algo \(R: \setof{0, 1}^* \to \setof{0, 1}^*\), s.t. \(\forall x: x \in A \iff R(x) \in B\).
\end{definition}

\begin{theorem}
  If \(A \leq_m^P B\), \(B \in \P\), then \(A \in \P\).
\end{theorem}

\begin{proof}
  Given poly-time \(M_B\) for \(B\), reduction \(R\), \(M_B(R(x))\) decides \(x \overset{?}{\in} A\) in \(\poly(|x|)\) time.
\end{proof}

\begin{theorem}
  \(\leq_m^P\) is transitive: if \(A \leq_m^P B, B \leq_m^P C\), then \(A \leq_m^P C\).
\end{theorem}

\begin{definition}
  \( \prob{4CHROMA} = \setof{\encoding{G} : \text{\(G\)'s chromatic number is \(4\)}} \).
\end{definition}

\begin{claim}
  \( \prob{4CHROMA} \leq_T^P \prob{SAT} \), where \(T\) stands for Turing, can use the subroutine many times.
\end{claim}

\begin{remark}
  Here can not use \( \phi = \phi_4 \wedge \neg \phi_3 \), since \(\neg \phi_3\) is satisfiable can not imply \(\phi_3\) is not satisfiable.
\end{remark}

\begin{theorem}
  If \(A \leq_T^P B\), \(B \in \P\), then \(A \in \P\).
\end{theorem}

\begin{theorem}
  If \(A \leq_m^P B\), \(B \in \NP\), then \(A \in \NP\).
\end{theorem}

\begin{remark}
  This theorem seemingly false for \(\leq_T^P\).
\end{remark}

\begin{theorem}
  \( \prob{CIRCUIT-SAT} \leq_m^P \prob{FORMULA-SAT} \).
\end{theorem}

\begin{theorem}
  \( \prob{CIRCUIT-SAT} \leq_m^P \prob{3-SAT} \).
\end{theorem}

upcoming: Cook-Levin Theorem

\begin{theorem}[Cook-Levin Theorem]
  \( \forall L \in \NP : L \leq_m^P \prob{CIRCUIT-SAT} \).
\end{theorem}

\section{Reading}

\subsection{Sipser 7.4 (\NP-completeness)}

\begin{enumerate}
  \item poly-time reductions (mapping reduction)
  \item Definition of NP-complete
\end{enumerate}

\end{document}
