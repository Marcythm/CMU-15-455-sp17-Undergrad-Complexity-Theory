\special{dvipdfmx:config z 0}
\documentclass{article}
\usepackage{marcythm}

\title{Undergraduate Complexity Theory \\ Lecture 6: Problems in \P}
\author{Marcythm}
% \date{\today}
\date{July 10, 2022}

\begin{document}
\maketitle{}

\section{Lecture Notes}

Recap: Time Hierarchy Theorem \(\implies\) \(\exists\) languages decidable in exponential time but not in \P.

\begin{definition}
  \( \EXP = \bigcup_{k \in \N} \TIME\parens{2^{n^k}} = \text{languages decidable in \(2^{\poly(n)}\) time} \).
\end{definition}

\begin{fact}
  \(\P \subsetneq \EXP\).
\end{fact}

\begin{enumerate}
  \item \( \prob{(ST-)PATH} \in \EXP \), and with a cleverer algo, \( \prob{PATH} \in \P \).
  \item \( \prob{2-COL} \in \EXP \), with a cleverer algo, \( \prob{2-COL} \in \P \) too.
  \item \( \prob{3-COL} \in \EXP \), but believed not in \P.
\end{enumerate}

\begin{theorem}[BE '05]
  \prob{3-COL} is in \(O(1.33^n)\) time.
\end{theorem}

\begin{enumerate}
  \item[4.] \( \prob{LCS} \in \EXP\), and with techniques like memorization and dynamic programming, \( \prob{LCS} \in \P \).
  \item[5.] \( \prob{3-CLIQUE} \in O(n^3) \).\ in \(O(n^2)\)? not sure.
\end{enumerate}

\begin{fact}
  can do \prob{3-CLIQUE} in \(O(n^3 / \log^2 n)\) time, or \(O(n^{2.38})\) time.
\end{fact}

\begin{enumerate}
  \item[6.] \( \prob{4-CLIQUE} \in \TIME(n^4) \).\ in \(\TIME(n^3)\)?
  \item[7.] \prob{k-CLIQUE} within brute force is time \(O\parens{\binom{n}{k} \binom{k}{2}} = O(n^k)\): \( \prob{k-CLIQUE} \in \EXP \).\ in \P? don't know.
\end{enumerate}

\section{Reading}

\subsection{Sipser 7.2 (The Class \P)}

\end{document}
