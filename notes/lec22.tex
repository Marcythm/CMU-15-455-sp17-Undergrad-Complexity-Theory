\special{dvipdfmx:config z 0}
\documentclass{article}
\usepackage{marcythm}

\title{Undergraduate Complexity Theory \\ Lecture 22: \BPP}
\author{Marcythm}
% \date{\today}
\date{July 24, 2022}

\begin{document}
\maketitle{}

\section{Lecture Notes}

\begin{definition}
  \BPP, bounded error probabilistic computation (two sided-error prob poly time), \(L \in \BPP\) if \(\exists\) PTM \(N\) s.t.
  \[ \begin{aligned}
    x \in L &\implies \Pr\bracks{\text{\(N(x)\) accept}} \geq 2/3 \\
    x \notin L &\implies \Pr\bracks{\text{\(N(x)\) accept}} \leq 1/3
  \end{aligned} \]
\end{definition}

current hierarchy: \(\P \subseteq \ZPP \subseteq \RP, \coRP \subseteq \RP \cup \coRP \subseteq \BPP\), all these are believed to be equal!

Alternative View: DTM \(M(x, r)\) with input \(x\) and a random tape input \(r\).

\begin{lemma}
  \(\BPP \subseteq \EXP\).
\end{lemma}

\begin{corollary}
  \(\BPP \subseteq \PSPACE\).
\end{corollary}

\(\BPP \subseteq \NP\)? not known. Even cannot separate \BPP from \NEXP.

\begin{theorem}
  \(\P = \NP \implies \P = \BPP\) (contra: \(\P \neq \BPP \implies \P \neq \NP\)).
\end{theorem}

\begin{definition}
  {\sf P/poly} is the class of languages with a circuit family of poly size deciding it.
\end{definition}

\begin{theorem}
  \(\BPP \subseteq {\sf P/poly}\).
\end{theorem}

\begin{proof}
  \(L \in \BPP\), \(\exists\) DTM \(M\), \(M(x, r)\) acc with \(p \geq 1 - 2^{-2|x|}\) if \(x \in L\), acc with \(p \leq 2^{-2|x|}\) if \(x \notin L\) in poly time.
  \(M\) can be translated into poly size circuit \(C_M\), which has two kinds of inputs, \(x\) and \(r\). The \(r\) input is what we should get rid of. \\
  For each fixed \(x \in \setof{0, 1}^n\), all but \(1/4^n\) of random coins \(r\) yield correct answer. Since only \(2^n\) possible \(x\) and \(1/4^n\) bad \(r\) for each, there are at most \(1/2^n\) \(r\)s are bad for some \(x\), i.e. most of \(r\)s are simultaneously good for all \(x\), so find them and hardwire them into circuit.
\end{proof}

Derandomization:

\begin{theorem}['98]
  If \prob{3SAT} requires circuit family of size \(2^{\delta n}\) for some \(\delta > 0\), then \P = \BPP.
\end{theorem}

two major steps: (worst-case hardness) to (strong average-case hardness) to (PRNG)

\section{Reading}

\subsection{sipser 10.2 (Probabilistic Algorithms)}

\subsubsection{Read-Once Branching Programs}

proof of \({\it EQ}_{\prob{ROBP}} \in \BPP\): construct polynomial, randomly select an element in finite field.

\end{document}
