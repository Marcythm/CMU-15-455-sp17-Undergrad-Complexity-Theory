\special{dvipdfmx:config z 0}
\documentclass{article}
\usepackage{marcythm}

\title{Undergraduate Complexity Theory \\ Lecture 19: From \P-Completeness to \PSPACE-Completeness}
\author{Marcythm}
% \date{\today}
\date{July 21, 2022}

\begin{document}
\maketitle{}

\section{Lecture Notes}

today: A \P-complete problem (under \(\leq_m^L\)) \(C\): \(C \in \L (\NL, \NC) \iff \P \subseteq \L (\NL, \NC)\).

Informally, \( \NC = \setof{L : \text{\(L\) solvable in \(\polylog(n)\) time, given \(\poly(n)\) processors}} \).

\begin{fact}
  \P-complete language: \prob{Horn-SAT}, \prob{Linear-Programming}, \prob{CIRCUIT-EVAL}
\end{fact}

\begin{fact}
  Empirically, whenever \(A \leq_m^P B\), it seems that \(A \leq_m^L B\) too.
\end{fact}

\begin{theorem}
  \( \forall A \in \NP: A \leq_m^L \prob{CIRCUIT-SAT} \leq_m^L \prob{3-SAT}\).
\end{theorem}

\begin{theorem}
  Cook-Levin Theorem is true even under \(\leq_m^L\).
\end{theorem}

sketch of proof: similar to Cook-Levin, each gadget can be constructed in logspace.

\begin{corollary}
  \prob{CIRCUIT-EVAL} is \P-complete.
\end{corollary}

\begin{definition}
  \prob{Q-SAT}, or \prob{TQBF}, where Q stands for quantified: Totally Quantified Boolean Formula.
  \[ \prob{TQBF} = \setof{\text{true sentences of form \(Q_1 x_1 Q_2 x_2 \ldots \phi(x)\) where \(Q_i \in \setof{\forall, \exists}\)}} \]
\end{definition}

\prob{FORMULA-SAT} is a variant where all \(Q_i\)s are \(\exists\).

\begin{proposition}
  \( \prob{TQBF} \in \PSPACE \).
\end{proposition}

use recursive algo with linear space and exp time.

\begin{theorem}
  \prob{TQBF} is \PSPACE-hard.
\end{theorem}

\begin{proof}
  similar to Cook-Levin, but use quantifiers to reduce size (on timestamp) from exp to poly.
\end{proof}

\begin{corollary}
  \prob{TQBF} is \PSPACE-complete.
\end{corollary}

\section{Reading}

\subsection{sipser 8.3 (\PSPACE completeness)}

definition of \PSPACE-complete: poly-time reduction.

\begin{remark}
  Whenever we define complete problems for a complexity class, the reduction model must be more limited than the model used for defining the class itself.
\end{remark}

definition of \prob{TQBF}: prenex normal form (all quantifiers appears in the beginning)

formula game, generalized geography, and reduction from \prob{GG} to \prob{TQBF}.

\end{document}
